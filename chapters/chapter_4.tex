\newpage
\begin{center}
  \textbf{\large 4. Эксперименты}
\end{center}
\refstepcounter{chapter}
\addcontentsline{toc}{chapter}{4. Эксперименты}

    В данном разделе приведено описание экспериментов.

    \section{Октрытие шкафа в симуляторе}

        \subsection{Описание среды}

            Манипулятору требуется открыть выдвижной ящик шкафа на заданное расстояние. В качестве наблюдений доступно следующее:

            \begin{itemize}
                \item Изображение с камеры $I$, расположенной на инструменте робота
                \item Информация о положениях сочленений робота $q$
                \item Информация о скоростях сочленений робота $\dot{q}$
            \end{itemize}

            Манипулятором является 7-ми осевой робот Franka Emika Panda. Эпизод считается успешным, если агент менее чем за 800 итераций, открывает выдвижной ящик на 15 сантиметров от исходного положения, изначально шкаф закрыт. 
            
            В начале каждого эпизода рандомизируются положения и ориентация шкафа, а также начальное положение сочленений робота. 

        \subsection{Процесс обучения}

            Для обучения агента был использован алгоритм имитационного обучения, а именно Behavioral Cloning (BC). Для обработки изображений извлекались признаки с помощью нейросетевой архитектуры ResNet18. Веса сети были взяты из IsaacLab и заморожены. 
            
            Было записано 1000 успешных экспертных демонстраций, каждая демонстрация содержала следующую информацию:~\begin{itemize}
                \item Признаки изображения $f(I)$
                \item Информация о положениях сочленений робота $q$
                \item Информация о скоростях сочленений робота $\dot{q}$
            \end{itemize}

            Хранение в демонстрациях не исходных изображений, а их признаков позволяет значительно уменьшить размер обучающего набора данных, снизить размер итоговой модели и заметно уменьшить время обучения. 

            Обучения проводилось на 2000 эпохах и заняло приблизительно 60 минут с использованием видеокарты NVIDIA GeForce RTX 4070 16 GB. По итогу обучения, на исходной среде доля успешных эпизодов агента составила примерно: $R_0 = 0.7$

            \subsection{Оценивание робастности модели}

                Согласно пункту \ref{method} была проведена валидация обученной модели на различных вариациях среды, результаты представлены на таблице \ref{res-Table1}. Для каждой вариации эксперимент проводился на 50 эпизодах ($N = 50$).

\begin{table}[h]
  \centering{
    \begin{tabular}{ | @{\hskip 5pt} p{3cm} @{\hskip 5pt} | @{\hskip 5pt} p{4.5cm} @{\hskip 5pt} | @{\hskip 5pt} p{4cm} @{\hskip 5pt} | @{\hskip 5pt} p{3cm} @{\hskip 5pt} | }
      \hline
      Рандомизация & Описание & Параметры распределения & Доля успешных эпизодов ($R_j$) \\ \hline 
     Отсутствие рандомизаций & Исходная среда & - & 0.7 \\ \hline
     \multicolumn{4}{|c|}{\textit{Визуальные параметры}} \\ \hline  
     Текстура ОМ & Рандомизированная текстура шкафа & $\mathrm{Uniform}(\mathcal{X}_{textures})$ & 0.0 \\ \hline
     Цвет ОМ & Рандомизированный цвет шкафа  & $\mathrm{Uniform}(0.0, 1.0)$ & 0.04 \\ \hline
     Текстура стен & Рандомизированная текстура стен & $\mathrm{Uniform}(\mathcal{X}_{textures})$ & 0.54 \\ \hline
     Цвет стен & Рандомизированный цвет стен & $\mathrm{Uniform}(0.0, 1.0)$ & 0.3 \\ \hline
     Цвет источника освещения & Рандомизированный цвет освещения & $\mathrm{Uniform}(0.0, 0.75)$ & 0.72 \\ \hline
      \multicolumn{4}{|c|}{\textit{Физические параметры}} \\ \hline  
       Масса объектов & Рандомизированная масса шкафа & $\mathrm{Uniform}(0.1, 4.1)$ & 0.62 \\ \hline 
       Трения объектов & Рандомизированная масса шкафа & $\mathrm{Uniform}(0.5, 1.5)$ & 0.34 \\ \hline
    \end{tabular}
  }
  \caption{Результаты валидации модели BC на задаче с отрытием шкафа}
  \label{res-Table1}
\end{table}
