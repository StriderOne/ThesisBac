\newpage
\begin{center}
  \textbf{\large 4. Методические рекомендации}
\end{center}
\refstepcounter{chapter}
\addcontentsline{toc}{chapter}{4. Методические рекомендации}

  На основе результатов экспериментов метода можно сформировать следующие рекомендации:

  \begin{itemize}
      \item Для получения валидной оценки робастности следует исследовать модели обладающие на исходной среде показателем эффективности $R_0 > 0.5$;

      \item Для определения параметров $\varepsilon$ и $\mathcal{X}$ стоит учитывать потенциальные внешние возмущения в целевом домене. Если в процессе работы агента среда может радикально отличаться от исходной, то стоит сделать параметр $\varepsilon$ больше, а соответствующие множества для дискретных рандомизаций -- разнообразнее; 

      \item При исследовании моделей, обученных на реальных данных, следует значительно уменьшить разрыв между симулятором и реальностью, для получения более точных результатов. Для этого можно воспользоваться инструментами, представленными в работе Simpler;

      \item Стоит аккуратно настраивать рандомизации для объекта манипулирования, так как даже модели с высокой обобщающей способностью могут испытывать трудности в решении задачи при радикальном изменении внешнего вида объекта;
      
      \item Результаты работы метода для прогнозирования эффективности модели в реальных условиях стоит рассматривать в одностороннем порядке. Если модель имеет большую эффективность на рандомизациях, то она с высокой вероятностью будет иметь большую эффективность и в реальных условиях. Однако, обратное может быть неверно. Если реальная среда не подвергается существенным изменениям, то модель, обученная на реальных данных, может иметь большую эффективность, но при этом не быть робастной. 

      
  \end{itemize}


