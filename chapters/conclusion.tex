\newpage
\begin{center}
  \textbf{\large ЗАКЛЮЧЕНИЕ}
\end{center}
\refstepcounter{chapter}
\addcontentsline{toc}{chapter}{ЗАКЛЮЧЕНИЕ}

В результате выполнения выпускной квалификационной работы был проведен анализ существующих методов валидации обученных моделей в симуляцинной среде. Выделены основные направления, на основе работ The Colosseum и Simpler был разработан метод оценивания робастности моделей машинного обучения. 

Идея разработанного метода заключается в определении показателей эффективности модели на различных вариациях среды, где вариации создаются за счет изменения визуальных и физических параметров исходной среды. Метод позволяет определить параметры среды, к которым исследуемая модель наиболее чувствительна, а также спрогнозировать эффективность модели при запуске в реальных условиях.  

Для проверки метода были проведены эксперименты для четырех моделей на двух задачах. В ходе каждого эксперимента были посчитаны показатели эффективности также оценка робастности исследуемых моделей на различных вариациях. Также была рассчитана корреляция эффективности моделей в симуляторе и в реальности, где данные о реальных экспериментов были взяты из работы Simpler. В результате эксперимента был определен коэффициент корреляции Пирсона $r = 0.897$, что говорит о хорошей способности метода предсказывать эффективность модели в реальных условиях. 

В данной работе продемонстрирована возможность прогнозирования эффективности модели в реальности, используя синтетические данные, что является перспективным и необходимым направлением для повсеместного использования моделей искусственного интеллекта на реальных робототехнических системах.

