\newpage
\begin{center}
  \textbf{\large 4. Методические рекомендации}
\end{center}
\refstepcounter{chapter}
\addcontentsline{toc}{chapter}{5. Методические рекомендации}

  На основе результатов экспериментов метода можно сформировать следующие рекомендации:

  \begin{itemize}
      \item Для получения валидной оценки робастности следует исследовать модели обладающие на исходной среде показателем эффективности $R_0 > 0.5$;

      \item Для определения параметров $\varepsilon$ и $\mathcal{X}$ стоит учитывать потенциальные внешние возмущения в целевом домене. Если в процессе работы агента среда может радикально отличаться от исходной, то стоит сделать параметр $\varepsilon$ больше, а соответствующие множества для дискретных рандомизаций -- разнообразнее; 

      \item При исследовании моделей, обученных на реальных данных, следует значительно уменьшить разрыв между симулятором и реальностью, для получения более точных результатов. Для этого можно воспользоваться инструментами, представленными в работе Simpler;

      \item Стоит аккуратно настраивать рандомизации для объекта манипулирования, так как даже модели с высокой обобщающей способностью могут испытывать трудности в решении задачи при радикальном изменении внешнего вида объекта. 
      

      
  \end{itemize}


