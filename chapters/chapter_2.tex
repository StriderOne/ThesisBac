\newpage
\begin{center}
  \textbf{\large 2. Методология}
\end{center}
\refstepcounter{chapter}
\addcontentsline{toc}{chapter}{2. Методология}
\label{method}
В данном разделе приводится выбор и описание параметров среды для рандомизации, а также методология оценки робастности моделей.

\section{Выбор параметров среды для варьирования}

    Как и в случае доменной рандомизации, задачей является получить достаточно разнообразное распределение вариаций сцен, такое что потенциальная сцена реального мира лежит в рамках этого распределения. Такой подход подразумевает вариацию множества параметров, как физических, так и визуальных. Далее представлено описание 8 выбранных параметров:

    \begin{enumerate}
        \item Параметры объекта манипулирования (ОМ). Объект манипулирования представляет собой целевой объект, с которым робот непосредственное взаимодействует или манипулирует. Вариации ОМ включают: цвет ОМ, текстура ОМ;

        \item Параметры заднего плана. К заднему плану относятся объекты, являющиеся фоновыми на изображениях, доступным агенту, к таким можно отнести стены и источник освещения. Вариации заднего плана включают: цвет стен, текстура стен, цвет источника освещения, яркость источника освещения;

        \item Физические параметры объектов. Вариации физических параметров включают: масса объектов, коэффициенты трения объектов.
    \end{enumerate}

    Выбранные вариации соответствуют потенциальным отличиям между исходным доменном и целевым, а также возмущениям, присущим среде реального мира. 

\section{Процедура варьирования параметров}
        \label{method_randomization}
    Метод варьирования параметров был заимствован также из доменной рандомизации \cite{Peng_2018}. В зависимости от природы параметра было выбрано 3 метода варьирования:

    \begin{enumerate}
        \item \textbf{Непрерывный скалярный случай.} Если параметр принимает некоторое скалярное значение $a \in R$, то его распределение может быть выбрано, как:
        \begin{equation}
            a \sim Uniform(a^* - \varepsilon, a^* + \varepsilon),
        \end{equation}
        где $a^*$ -- значение параметра для исходной, не рандомизированной среды; $\varepsilon$ --  настраиваемый параметр. К параметрам этого типа можно отнести коэффициенты трения, массу. Возможно ситуация, когда значение параметра $a^*$ не известно или не доступно, например, если изучаемая модель была обучена на реальных данных. Для решения этой проблемы есть несколько способов:
            
            \begin{itemize}
                \item Некоторым образом идентифицировать параметры реального мира, то есть получить оценку интересующего параметра $\hat{a}^*$
                \item Зафиксировать приблизительное значение параметра $\hat{a}^*$ и выбрать достаточно большое значение для $\varepsilon$, чтобы распределение вариаций с большей вероятностью покрывало целевой домен.
            \end{itemize}

        \item \textbf{Непрерывный многомерный случай.} Если параметр принимает не скалярное значение, а некоторый вектор размера $n$: $x \in \mathcal{R}^n$, то распределение этого вектора может быть выбрано, как: 

            \begin{equation}
                x_i \sim \mathrm{Uniform}(x_i^* - \varepsilon, x_i^* + \varepsilon), \quad \forall i \in {1, \dots, n},
            \end{equation}

        где $x_i^*$ -- значение элемента вектора для исходной, не рандомизированной среды; $\varepsilon$ --  настраиваемый параметр, который может быть представлен, как вектор размера 3. Ситуация, когда значения $x_i^*$ неизвестны, решаются аналогично предыдущему случаю.

        \item \textbf{Дискретный случай.} Некоторые параметры не могут быть представлены в непрерывном пространстве. Пусть параметр принимает некоторую сущность $x \in \mathcal{X}$, где $\mathcal{X}$ - ограниченное множество допустимых вариаций, тогда распределение для такого параметра может быть представлено в виде:
        
        \begin{equation}
            x \sim \mathrm{Uniform}(\mathcal{X})
        \end{equation}

        К этому типу параметром, например, относятся текстуры, тогда в качестве $x$ выступает конкретная текстура, то есть изображение, равновероятно выбранная из заданного множества текстур $\mathcal{X}$. 
        
    \end{enumerate}

    Настраиваемые параметры, такие как $\varepsilon$ для непрерывных случаев или элементы множества $\mathcal{X}$ для дискретного случая, зависят от задачи, отличий между исходным и целевым доменами, а также от природы возможных возмущений целевого домена. 

    

\section{Метод оценивания робастности моделей}
        \label{method_robust}
    Для получения оценки робастности модели долю успешных эпизодов, полученную в результате эксперимента с исследуемой моделью. 
    
    Процедуру оценивая можно описать следующим образом:

    \begin{enumerate}
        \item Агент выполняет задачу в исходной, не рандомизированной среде. Проводится $N$ экспериментов, доля успешных эпизодов усредняется по количеству эпизодов:

        \begin{equation}
            R_0 = \frac{1}{N} \sum_{i=1}^N \mathbb{I}_{\text{succ}}(E^0_i)
        \end{equation}

        \item Для каждой возможной вариации среды агент аналогично выполняет задачу $N$ экспериментов. Доля успешных эпизодов усредняется по количеству эпизодов:

        \begin{equation}
            R_j = \frac{1}{N} \sum_{i=1}^N \mathbb{I}_{\text{succ}}(E^j_i), \quad \forall j \in {1, \dots, M}
        \end{equation}

        \item Метрика робастности модели вычисляется как разница между исходной долей успешных эпизодов $p_0$ и долей успешных эпизодов на рандомизированной среде $p_j$:  
        \begin{equation}  
            \Delta R_j = max(R_0 - R_j, 0),  
        \end{equation}  
        где $\Delta R_j$ характеризует робастность модели к $j$-му типу вариаций. Таким образом, чем ближе значение $\Delta R_j$ к $0$, тем устойчивее модель к вариации $j$.

        \item Итоговая метрика робастности модели может быть рассчитана, как усредненное значение $\Delta R_j$ по всем вариациям:

        \begin{equation}  
            \Delta R = \frac{1}{M} \sum_{j=1}^M \Delta R_j,
        \end{equation}  

        где $\Delta R$ характеризует общую робастность модели, аналогично, чем ближе значение $\Delta R$ к $0$, тем модель робастнее.
        
    \end{enumerate}

\section{Метод оценивания корреляции между эффективностью в симуляторе и в реальности}
    
    Чтобы оценить корреляцию эффективности модели в симуляторе и в реальности, был выбран метод из работы SIMPLER \cite{li24simpler}. Стандартным подходом для определения корреляции между двумя величинами является коэффициент корреляции Пирсона \cite{Pearson}, который может быть рассчитан как:

    \begin{equation}
    r_{XY} = \frac{\text{cov}(X, Y)}{\sigma_X \sigma_Y} = \frac{\sum_{i=1}^n (X_i - \bar{X})(Y_i - \bar{Y})}{\sqrt{\sum_{i=1}^n (X_i - \bar{X})^2} \sqrt{\sum_{i=1}^n (Y_i - \bar{Y})^2}},
    \end{equation}
    где:
    \begin{description}
        \item $X, Y$ -- рассматриваемые выборки
        \item $\text{cov}(X, Y)$ -- ковариация между $X$ и $Y$
        \item $\bar{X}, \bar{Y}$ -- выборочные средние
    \end{description}
    
    Коэффициент Пирсона позволяет оценивать степень линейной зависимости между переменными и применялся в предыдущих исследованиях для оценки качества имитационных оценочных систем \cite{Kadian_2020} путем измерения корреляции между показателями эффективности в реальных условиях и в симуляции. Высокий коэффициент корреляции Пирсона $(\simeq 1)$ свидетельствует о хорошо функционирующей имитационной оценочной системе, где улучшение показателей успешности в реальных условиях соответствует линейному росту успешности в симуляции. Напротив, более низкий коэффициент корреляции может указывать на слабую взаимосвязь между результатами оценки в реальных условиях и в симуляции. 

    
    
    