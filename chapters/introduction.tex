\newpage
\renewcommand{\contentsname}{\centerline{\large СОДЕРЖАНИЕ}}
\tableofcontents

\newpage
\begin{center}
  \textbf{\large ВВЕДЕНИЕ}
\end{center}
\addcontentsline{toc}{chapter}{ВВЕДЕНИЕ}

Современные модели машинного обучения демонстрируют высокую эффективность в различных робототехнических задачах, 
но их применение в реальных условиях остается ограниченным, так как среда в реальном мире обычно отличается от той, 
что была использована при обучении. Это делает актуальным вопрос разработки более робастных методов и проведения более полного тестирования обученных моделей.

Однако большинство существующих исследований проводят валидацию моделей в условиях, близких к обучающим, 
что не позволяет сделать выводы, о способности таких моделей адаптироваться под новые условия окружающей среды. 
В качестве решения описанной проблемы, могут выступать методы анализа робастности моделей, позволяющие осуществлять валидацию с варьированием параметров среды.

Целью настоящей работы является разработка и исследование метода анализа робастности обученных моделей к изменениям параметров среды, 
способного выявлять параметры среды, к которым модели наиболее чувствительны, а также прогнозировать эффективность моделей в реальных условиях. В качестве основы для анализа используются обученные модели, 
которые тестируются в симуляционной среде с варьируемыми параметрами, такими как освещение, текстуры, физические свойства объектов и другие факторы.
Результаты работы могут быть использованы для безопасного развертывания моделей на реальных робототехнических системах, а также для разработки более робастных моделей, способных эффективно функционировать в изменяющихся условиях.