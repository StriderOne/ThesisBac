\newpage
\begin{center}
  \textbf{\large 3. Разработка и реализация}
\end{center}
\refstepcounter{chapter}
\addcontentsline{toc}{chapter}{3. Разработка и реализация}

В данном разделе описан разработанный метод для оценивания робастности модели.

\section{Архитектура предлагаемого решения}

    Предлагаемый метод оценки робастности обученных моделей реализован в виде библиотеки на яызыке Python, основанного на фреймворке IsaacLab. Архитектура решения включает следующие ключевые компоненты:
        \begin{itemize}
            \item \textbf{VariationManagerCfg} - конфигурационный класс, определяющий вариаций среды и их параметры. Является уникальным для каждой среды и требует ручного создания. 
            \item \textbf{VariationManager} – основной модуль, ответственный за управление вариациями среды по заданному конфигурационному файлу VariationManagerCfg.
            \item \textbf{Validation} - класс, отвечающий за проведение экспериментов и обработку результатов.
        \end{itemize}

\section{Модуль варьирования параметров среды}
    Для каждой желаемой вариации создается экземпляр конфигурационного файла \textit{VariationCfg}, который содержит всю необходимую информацию для применения соответствующей рандомизации, он содержит следующую информацию:

    \begin{itemize}
        \item Название вариации для отражения в отчетных материалах
        \item Объект функции, которая непосредственно выполняет рандомизацию
        \item Момент, когда должна применяться рандомизация (при запуске симулятора, в начале нового эпизода и т.д.)
        \item Дополнительные параметры, в основном задающие распределение для рандомизированного параметра, распределение задается согласно пункту \ref{method_randomization}
    \end{itemize} 

    Чтобы хранить описания всех вариаций, используется класс \\ \textit{VariationManagerCfg}, который является опорным для класса \textit{VariationManager}, непосредственно управляющего всеми вариациями. 

    На нижнем уровне, ключевой является функция, применяющая рандомизацию. Если вариация подразумевает рандомизацию физических параметров, то обновленные параметры среды напрямую записываются в свойства соответствующих объектов и учитываются уже вначале следующего эпизода. 

    Вариация визуальных составляющих является более сложным процессом, ввиду особенностей работы симулятора Isaac Sim. Для реализации подобного функционала в симуляторе представлен фреймворк Omniverse Replicator, предназначенный для разработки пользовательских конвейеров генерации синтетических данных и сопутствующих сервисов. Данный фреймворк активно используется при реализации доменной рандомизации в Isaac Sim и поэтому является наиболее уместным для модуля варьирования параметров среды.
    
\section{Реализация алгоритма оценивания робастности модели}

    Для оценки робастности, для каждой вариации конфигурация среды изменяется согласно настройкам установленным в классе VariationManager. Далее оцениваемая модель запускается в среде с обновленной конфигурацией, включающую соответсвующую рандомизацию. Для каждой вариации подсчитывается доля успешных эпизодов и оценка робастности модели согласно пункту \ref{method_robust}. 
    